%% Preamble

\documentclass[11pt, a4paper, titlepage]{article}
\usepackage[utf8x]{inputenc}    % UTF-8 encoding
\usepackage[spanish]{babel} % culturallydetermined typographical (and other) rules

%% Fonts
\usepackage{fourier}
\usepackage[scaled=0.83]{helvet} %% Helvetica queda demasiado grande
% \renewcommand{\ttdefault}{txtt}
% \usepackage[scaled=0.9]{inconsolata}
\usepackage[scaled=0.75]{beramono}

\usepackage{hyperref}
\usepackage{parskip}            % Paragraph allingment
\usepackage{color}

% Macros for comments and todos
\usepackage[textsize=footnotesize, textwidth=2\marginparwidth]{todonotes}
\newcommand{\mr}[1]{\todo{MR: #1}}

%% Exercises
\usepackage[printsolution=true]{exercises}

%% Code
\usepackage{listings}
\lstset{basicstyle=\tt}

%% Sets of numbers
\usepackage{amssymb}

%% Drawing diagrams
\usepackage[all,cmtip]{xypic}

%% End Preamble

%% Begin document

\begin{document}

\title{Tarea 3}
\date{23 de Febrero de 2018}
\maketitle
\newpage



\section*{Ejercicio 5.2}

\begin{lstlisting}[language=pascal]
  proc fibonacci(x : int) returns int{
    if (x<=1) then return x;
    var res: int;
    var resAnterior: int;
    var i: int;
    var n: int;
    n = x-1;
    res = 1;
    for i = 2 to n do {
      var temp: int;
      temp = res;
      res = res + resAnterior;
      resAnterior = temp;
    }
    return res;
  }
\end{lstlisting}
\subsection*{Solución}

\section*{Ejercicio 5.17}
\subsection*{Solución}
La rutina del ejercicio calcula una función parcial, es decir, no está definida para todos
los elementos del conjunto $\mathbb{N}$, dado que solo funciona para el número 0.

La solución de una Máquina de Turing que calcule lo mismo que esta función sería la siguiente:

\[
\xymatrix{
  >R \ar[rr]^{0} \ar[dr]^{\_} && L_\rhd \\
  & M_* \ar@(dr,dl)
}
\]


\end{document}
