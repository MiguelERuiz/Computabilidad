%% Preamble

\documentclass[11pt, a4paper, titlepage]{article}
\usepackage[utf8x]{inputenc}    % UTF-8 encoding
\usepackage[spanish]{babel} % culturallydetermined typographical (and other) rules

%% Fonts
\usepackage{fourier}
\usepackage[scaled=0.83]{helvet} %% Helvetica queda demasiado grande
% \renewcommand{\ttdefault}{txtt}
% \usepackage[scaled=0.9]{inconsolata}
\usepackage[scaled=0.75]{beramono}

\usepackage{hyperref}
\usepackage{parskip}            % Paragraph allingment
\usepackage{color}

% Macros for comments and todos
\usepackage[textsize=footnotesize, textwidth=2\marginparwidth]{todonotes}
\newcommand{\mr}[1]{\todo{MR: #1}}

%% Exercises
\usepackage[printsolution=true]{exercises}

%% Math symbols
\usepackage{amssymb}

%% Code
\usepackage{listings}
\lstset{basicstyle=\tt}

%% Math equations
\usepackage{amsmath}

%% End Preamble

%% Begin document

\begin{document}

\title{Tarea 7}
\date{\today}
\maketitle
\newpage

\section*{Ejercicio 10.11}
\subsection*{Solución}

La línea 7 es código muerto; para justificarlo hay que analizar la línea 5 que
corresponde al if-then-else: si se cumple la condición del if, a la variable x
se le asignará el valor de j, que es 10. Cuando se ejecute la línea 6, correspondiente
a la condición del bucle while, ésta no se cumplirá y no se entrará dentro del bucle
while. Por el contrario, si se entra en la rama else, la variable x tendrá asignada
el valor de i, y tampoco entrará dentro del bucle while ya que cualquier número
que sea menor o igual que 0 no es mayor que 10.

\section*{Ejercicio 11.4}
\subsection*{Solución}
%% $$[pos] [-] [pos] = [¿?]$$
%% $$[pos] [-] [neg] = [pos]$$
%% $$[neg] [-] [pos] = [neg]$$
%% $$[neg] [-] [neg] = [¿?]$$
%% $$[¿?] [-] [0] = [¿?]$$
%% $$[0] [-] [¿?] = [¿?]$$
\begin{align*}
  [0] [-] [0] = [0]\\
  [pos] [-] [neg] = [pos]\\
  [neg] [-] [pos] = [neg]\\
  [pos] [-] [0] = [pos]\\
  [neg] [-] [0] = [neg]\\
  [0] [-] [neg] = [pos]\\
  [0] [-] [pos] = [neg]\\
  [neg] [-] [neg] = [\text{¿?}]\\
  [pos] [-] [pos] = [\text{¿?}]\\
  [\text{¿?}] [-] [0] = [\text{¿?}]\\
  [0] [-] [\text{¿?}] = [\text{¿?}]
\end{align*}
\newline
\section*{Ejercicio 11.9}
\subsection*{Solución}
La solución a este ejercicio está hecha con mi solución al ejercicio 11.4:

$$(2-1) * (5-5)$$
$$[pos] [-] [pos] * [pos] [-] [pos]$$
$$[\text{¿?}] [*] [\text{¿?}]$$
$$[\text{¿?}]$$

\section*{Ejercicio 11.11}
\subsection*{Solución}
Para la operación aritmética aproximada del cuadrado [2] se han definido los
siguientes casos:
\newline
\begin{align*}
  [pos]^{[2]} = [pos]\\
  [neg]^{[2]} = [pos]\\
  [0]^{[2]} = [0]\\
  [\text{¿?}]^{[2]} = [\text{¿?}]
\end{align*}
Por tanto, la solución para este ejercicio sería:
\begin{align*}
  [n-neg][+]([n-0]^{[2]}[+][pos]) & = & [n-neg][+]([pos]+[pos])
  \\
  & = & [n-neg][+][pos]
  \\
  & = &[pos]
\end{align*}
\section*{Ejercicio 11.12}
\subsection*{Solución}
\begin{align*}
  ([n-neg][−][n-pos])[+][\text{¿?}]^{[2]} & = & [\text{¿?}][+][\text{¿?}]^{[2]}
  \\
  & = & [\text{¿?}][+][\text{¿?}]
  \\
  & = & [\text{¿?}]
\end{align*}
\end{document}
