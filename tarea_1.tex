%% Preamble

\documentclass[11pt, a4paper, titlepage]{article}
\usepackage[utf8x]{inputenc}    % UTF-8 encoding
\usepackage[spanish]{babel} % culturallydetermined typographical (and other) rules

%% Fonts
\usepackage{fourier}
\usepackage[scaled=0.83]{helvet} %% Helvetica queda demasiado grande
% \renewcommand{\ttdefault}{txtt}
% \usepackage[scaled=0.9]{inconsolata}
\usepackage[scaled=0.75]{beramono}

\usepackage{hyperref}
\usepackage{parskip}            % Paragraph allingment
\usepackage{color}

% Macros for comments and todos
\usepackage[textsize=footnotesize, textwidth=2\marginparwidth]{todonotes}
\newcommand{\mr}[1]{\todo{MR: #1}}

%% Exercises
\usepackage[printsolution=true]{exercises}

%% Drawing Turing machines (automata in general)
\usepackage{tikz, wasysym}
\usetikzlibrary{automata,positioning}

%% End Preamble

%% Begin document

\begin{document}

\title{Tarea 1}
\date{12 de Febrero de 2018}
\maketitle
\newpage



\section{Ejercicio 3.2}
\subsection*{Solución}

\section{Ejercicio 3.3}
\subsection*{Solución}

Dado que la función de transición de la MT se representaría de la siguiente manera:

\begin{center}
  \begin{tabular}{| l | l | l | l |}
    \hline
    E. actual & Símbolo & Acción & E. nuevo \\ \hline
    0 & \_ & $\rightarrow$ & 1 \\ \hline
    1 & \_ & $\rightarrow$ & 2 \\ \hline
    1 & a & $\rightarrow$ & 1 \\ \hline
    2 & \_ & $\rightarrow$ & 2 \\ \hline
    2 & a & $\rightarrow$ & 3 \\ \hline
    3 & \_ & $\leftarrow$ & 9 \\ \hline
    3 & a & $\rightarrow$ & 5 \\ \hline
    4 & $\rhd$ & $\rightarrow$ & 10 \\ \hline
    4 & * & $\leftarrow$ & 4 \\ \hline
    5 & \_ & $\leftarrow$ & 6 \\ \hline
    5 & a & $\rightarrow$ & 5 \\ \hline
    6 & \_ & $\leftarrow$ & 7 \\ \hline
    6 & a & \_ & 6 \\ \hline
    7 & \_ & $\leftarrow$ & 8 \\ \hline
    7 & a & $\leftarrow$ & 7 \\ \hline
    8 & \_ & $\leftarrow$ & 8 \\ \hline
    8 & a & \_ & 2 \\ \hline
    9 & \_ & $\leftarrow$ & 4 \\ \hline
    9 & a & \_ & 9 \\
    \hline
  \end{tabular}
\end{center}


el diagrama para representar el MT sería el siguiente:


\begin{tikzpicture}[shorten >=1pt,node distance=2cm,on grid,auto]
  \node[state, initial] (0) {$s$};
  \node[state] (1) [right=of 0] {$q_0$};
  \node[state] (2) [right=of 1] {$q_1$};
  \node[state] (3) [below=of 2] {$q_2$};
  \node[state] (5) [right=of 3] {$q_4$};
  \node[state] (6) [right=of 5] {$q_5$};
  \node[state] (7) [above=of 6] {$q_6$};
  \node[state] (8) [above=of 7] {$q_7$};
  \node[state] (9) [below=of 3] {$q_8$};
  \node[state] (4) [left=of 9] {$q_3$};
  \node[state, accepting] (10) [left=of 4] {$q_9$};
  \path[->]
  (0) edge node {$\_\:\_\:R$} (1)
  (1) edge [loop above] node {$a\:a\:R$} (1)
  (1) edge node {$\_\:\_\:R$} (2)
  (2) edge node {$a\:a\:R$} (3)
  (2) edge [loop above] node {$\_\:\_\:R$} (2)
  (3) edge node {$a\:a\:R$} (5)
  (5) edge [loop below] node {$a\:a\:R$} (5)
  (5) edge node {$\_\:\_\:L$} (6)
  (6) edge [loop right] node {$a\:\_$} (6)
  (6) edge node {$\_\:\_\:L$} (7)
  (7) edge [loop right] node {$a\:a\:L$} (7)
  (7) edge node {$\_\:\_\:L$} (8)
  (8) edge [loop right] node {$\_\:\_\:L$} (8)
  (8) edge node {$a\:\_\:L$}(2)
  (3) edge node {$\_\:\_\:L$} (9)
  (9) edge [loop below] node {$a\:\_$} (9)
  (9) edge node  {$\_\:\_\:L$} (4)
  (4) edge [loop below] node {$*\:*\:L$}
  (4) edge node {$\rhd\:\rhd\:R$} (10);
\end{tikzpicture}

\section{Ejercicio 3.13}
\subsection*{Solución}

Para poder la solución del lenguaje decidido por la MT, se va analizar paso por
paso el funcionamiento de la misma:

Al principio, se desplaza a la derecha una casilla:
\begin{itemize}
\item Si el símbolo leído es \_, el lenguaje es aceptado.
\item Si el símbolo es una d, se desplaza una casilla a la derecha.
\item Si el símbolo es b o c, el lenguaje es rechazado.
\item Si el símbolo leído es una a, escribe una d en dicha casilla
  y vuelve a moverse una casilla hacia la derecha:
  \begin{itemize}
  \item Si el símbolo leído es a o d, se desplaza una casilla a la derecha.
  \item Si el símbolo leído es c o \_, el lenguaje es rechazado.
  \item Si el símbolo leído es una b, escribe una d en dicha casilla
    y se desplaza una casilla a la derecha:
    \begin{itemize}
    \item Si el símbolo leído es b o d, se desplaza una casilla hacia la derecha.
    \item Si el símbolo leído es a o \_, el lenguaje es rechazado.
    \item Si el símbolo leído es c, se escribe una d en la casilla correspondiente
      y se desplaza hacia la izquierda hasta encontrar un \_, y luego vuelve al
      estado inicial.
    \end{itemize}
  \end{itemize}
\end{itemize}
Por tanto, el lenguaje decidido es el conjunto \{a^nb^nc^n , n>= 0\}
%% End document
\end{document}
