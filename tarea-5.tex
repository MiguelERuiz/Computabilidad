%% Preamble

\documentclass[11pt, a4paper, titlepage]{article}
\usepackage[utf8x]{inputenc}    % UTF-8 encoding
\usepackage[spanish]{babel} % culturallydetermined typographical (and other) rules

%% Fonts
\usepackage{fourier}
\usepackage[scaled=0.83]{helvet} %% Helvetica queda demasiado grande
% \renewcommand{\ttdefault}{txtt}
% \usepackage[scaled=0.9]{inconsolata}
\usepackage[scaled=0.75]{beramono}

\usepackage{hyperref}
\usepackage{parskip}            % Paragraph allingment
\usepackage{color}

% Macros for comments and todos
\usepackage[textsize=footnotesize, textwidth=2\marginparwidth]{todonotes}
\newcommand{\mr}[1]{\todo{MR: #1}}

%% Exercises
\usepackage[printsolution=true]{exercises}

%% Math symbols
\usepackage{amssymb}

%% Code
\usepackage{listings}
\lstset{basicstyle=\tt}

%% End Preamble

%% Begin document

\begin{document}

\title{Tarea 5}
\date{\today}
\maketitle
\newpage

\section*{Ejercicio 7.9}
\subsection*{Solución}

\section*{Ejercicio 8.9}
\subsection*{Solución}
\begin{lstlisting}[language=pascal]
  proc p(n:int) retunrs int{
    var i:int, j:int;
    if (n+4 > 10 ^ 4-n < 1) then{
      j:=0;
    }else{
      j:=n+8;
    }
    i:=0;
    j:=0;
    while(i<6) do{
      j:=j+n;
      i:=i+1;
    }
    return j;
  }
\end{document}
