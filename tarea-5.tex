%% Preamble

\documentclass[11pt, a4paper, titlepage]{article}
\usepackage[utf8x]{inputenc}    % UTF-8 encoding
\usepackage[spanish]{babel} % culturallydetermined typographical (and other) rules

%% Fonts
\usepackage{fourier}
\usepackage[scaled=0.83]{helvet} %% Helvetica queda demasiado grande
% \renewcommand{\ttdefault}{txtt}
% \usepackage[scaled=0.9]{inconsolata}
\usepackage[scaled=0.75]{beramono}

\usepackage{hyperref}
\usepackage{parskip}            % Paragraph allingment
\usepackage{color}

% Macros for comments and todos
\usepackage[textsize=footnotesize, textwidth=2\marginparwidth]{todonotes}
\newcommand{\mr}[1]{\todo{MR: #1}}

%% Exercises
\usepackage[printsolution=true]{exercises}

%% Math symbols
\usepackage{amssymb}

%% Code
\usepackage{listings}
\lstset{basicstyle=\tt}

%% End Preamble

%% Begin document

\begin{document}

\title{Tarea 5}
\date{\today}
\maketitle
\newpage

\section*{Ejercicio 7.9}
\subsection*{Solución}
La solución que he seguido ha sido utilizar el método de la diagonalización:
partiendo en la siguiente tabla, en donde las columnas estarían los números
naturales desde 0 a n y en las filas estarían las funciones desde f(0) a f(n)
tal y como se muestra a continuación.

\begin{center}
  \begin{tabular}{|c c c c c c c c c c c c|}
    \hline
    \, & 0 & 1 & 2 & 3 & 4 & 5 & 6 & 7 & 8 & 9 & \ldots \\ \hline
    f(0) & 0 & 1 & 0 & 0 & 0 & 0 & 0 & 0 & 0 & 0 & \ldots \\ \hline
    f(1) & 1 & 0 & 0 & 0 & 0 & 0 & 0 & 0 & 0 & 0 & \ldots \\ \hline
    f(2) & 0 & 0 & 1 & 0 & 0 & 0 & 0 & 0 & 0 & 0 & \ldots \\ \hline
    f(3) & 0 & 0 & 0 & 0 & 1 & 0 & 0 & 0 & 0 & 0 & \ldots \\ \hline
    f(4) & 0 & 0 & 0 & 0 & 1 & 0 & 0 & 0 & 0 & 0 & \ldots \\ \hline
    f(5) & 0 & 0 & 0 & 0 & 1 & 0 & 0 & 0 & 0 & 0 & \ldots \\ \hline
    f(6) & 0 & 0 & 0 & 0 & 0 & 0 & 0 & 1 & 0 & 0 & \ldots \\ \hline
    f(7) & 0 & 0 & 0 & 0 & 0 & 0 & 0 & 1 & 0 & 0 & \ldots \\ \hline
    f(8) & 0 & 0 & 0 & 0 & 0 & 0 & 0 & 0 & 0 & 1 & \ldots \\ \hline
    f(9) & 0 & 0 & 0 & 0 & 0 & 0 & 0 & 0 & 1 & 0 & \ldots \\ \hline
    \ldots
    
  \end{tabular}
\end{center}

%% \begin{center}
%%   \begin{tabular}{| l | l | l | l |}
%%     \hline
%%     E. actual & Símbolo & Acción & E. nuevo \\ \hline
%%     0 & \_ & $\rightarrow$ & 1 \\ \hline
%%     1 & \_ & $\rightarrow$ & 2 \\ \hline
%%     1 & a & $\rightarrow$ & 1 \\ \hline
%%     2 & \_ & $\rightarrow$ & 2 \\ \hline
%%     2 & a & $\rightarrow$ & 3 \\ \hline
%%     3 & \_ & $\leftarrow$ & 9 \\ \hline
%%     3 & a & $\rightarrow$ & 5 \\ \hline
%%     4 & $\rhd$ & $\rightarrow$ & 10 \\ \hline
%%     4 & * & $\leftarrow$ & 4 \\ \hline
%%     5 & \_ & $\leftarrow$ & 6 \\ \hline
%%     5 & a & $\rightarrow$ & 5 \\ \hline
%%     6 & \_ & $\leftarrow$ & 7 \\ \hline
%%     6 & a & \_ & 6 \\ \hline
%%     7 & \_ & $\leftarrow$ & 8 \\ \hline
%%     7 & a & $\leftarrow$ & 7 \\ \hline
%%     8 & \_ & $\leftarrow$ & 8 \\ \hline
%%     8 & a & \_ & 2 \\ \hline
%%     9 & \_ & $\leftarrow$ & 4 \\ \hline
%%     9 & a & \_ & 9 \\
%%     \hline
%%   \end{tabular}
%% \end{center}


\section*{Ejercicio 8.9}
\subsection*{Solución}
\begin{lstlisting}[language=pascal]
  proc p(n:int) retunrs int{
    var i:int, j:int;
    if (n+4 > 10 ^ 4-n < 1) then{
      return 0;
    }else{
      return n+8;
    }
    i:=0;
    j:=0;
    while(i<6) do{
      j:=j+n;
      i:=i+1;
    }
    return j;
  }
\end{lstlisting}
\end{document}
