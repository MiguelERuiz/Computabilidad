%% Preamble

\documentclass[11pt, a4paper, titlepage]{article}
\usepackage[utf8x]{inputenc}    % UTF-8 encoding
\usepackage[spanish]{babel} % culturallydetermined typographical (and other) rules

%% Fonts
\usepackage{fourier}
\usepackage[scaled=0.83]{helvet} %% Helvetica queda demasiado grande
% \renewcommand{\ttdefault}{txtt}
% \usepackage[scaled=0.9]{inconsolata}
\usepackage[scaled=0.75]{beramono}

\usepackage{hyperref}
\usepackage{parskip}            % Paragraph allingment
\usepackage{color}

% Macros for comments and todos
\usepackage[textsize=footnotesize, textwidth=2\marginparwidth]{todonotes}
\newcommand{\mr}[1]{\todo{MR: #1}}

%% Exercises
\usepackage[printsolution=true]{exercises}

%% End Preamble

%% Begin document

\begin{document}

\title{Tarea 4}
\subtitle{Un punto de vista que pone en duda la Tesis de Church}
\date{\today}
\maketitle

El punto de vista que presento\footnote{\url{https://journals.openedition.org/philosophiascientiae/772}}
es el de la profesora Carol E. Cleland, en el cual argumenta la falsedad de la
Tesis de Church, partiendo del hecho de que los límites de la computación son físicos y no lógicos.
\newpage


\end{document}
