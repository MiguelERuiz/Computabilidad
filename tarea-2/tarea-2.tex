%% Preamble

\documentclass[11pt, a4paper, titlepage]{article}
\usepackage[utf8x]{inputenc}    % UTF-8 encoding
\usepackage[spanish]{babel} % culturallydetermined typographical (and other) rules

%% Fonts
\usepackage{fourier}
\usepackage[scaled=0.83]{helvet} %% Helvetica queda demasiado grande
% \renewcommand{\ttdefault}{txtt}
% \usepackage[scaled=0.9]{inconsolata}
\usepackage[scaled=0.75]{beramono}

\usepackage{hyperref}
\usepackage{parskip}            % Paragraph allingment
\usepackage{color}

% Macros for comments and todos
\usepackage[textsize=footnotesize, textwidth=2\marginparwidth]{todonotes}
\newcommand{\mr}[1]{\todo{MR: #1}}

%% Exercises
\usepackage[printsolution=true]{exercises}

%% Drawing Turing machines (automata in general)
\usepackage{tikz, wasysym}
\usetikzlibrary{automata,positioning}


%% Drawing diagrams
\usepackage[all,cmtip]{xypic}

%% Sets of numbers
\usepackage{amssymb}

%% visible white space
\newcommand\vartextvisiblespace[1][.5em]{%
  \makebox[#1]{%
    \kern.07em
    \vrule height.3ex
    \hrulefill
    \vrule height.3ex
    \kern.07em
  }% <-- don't forget this one!
}

%% End Preamble

%% Begin document

\begin{document}

\title{Tarea 2}
\date{18 de Febrero de 2018}
\maketitle
\newpage



\section*{Ejercicio 3.15}
\subsection*{Solución}


Para saber si el conjunto enunciado es o no decidible, vamos a hacer uso de las
deficiones que tienen que ver con definir un lenguaje:

Una MT \textit{M} decide un lenguaje \textit{L} si $\forall x \in \Sigma^*:$
\begin{itemize}
  \item Si x $\in \textit{L}, \textit{M}$ termina en Y.
  \item Si x $\notin \textit{L}, \textit{M}$ termina en N.
\end{itemize}

Un lenguaje es decidible o recursivo si $\exists \textit{M}$ que lo decide.


Por tanto, para poder demostrar que el conjunto \textit{B} es decidible, tenemos
que encontrar una MT \textit{M} que lo decida. Antes de definir la MT, vamos a deducir
qué números pertenecen al conjunto $B$. Dada la definicón matemática del conjunto, el
primer número perteneciente a $B$ sería el número 10 en notación decimal, dado
que todos los números por debajo de él (desde el 1 hasta el 9) son menores que él y su
$c(m)$ es menor que $c(10)$. Por tanto, los números pertencientes a este conjunto son
todas las potencias de 10 que pertencen al conjunto $\mathbb{N}$, el 1 excluído.

Así pues, la MT que decide al conjunto $B$ sería:


\[
\xymatrix{
  >RR \ar[r]^{1} \ar[dr]^{\_} & R \ar[r]^{0} \ar[d]^{\_} & R \ar[ld]^{\_} \ar@(ur,ul)^{0} \ar[r]^{\vartextvisiblespace} & Y \\
  & N
}
\]

\section*{Ejercicio 3.20}
\subsection*{Solución}

Para la primera parte del ejercicio, definimos la MT $M_4$ de la siguiente
manera:

$ M_4 = < K, \Sigma, s, H, \delta >$

donde
\begin{itemize}
\item $ K = \{ s, q_0, q_1, q_2, q_3, q_4, q_5, q_6, q_7, h, e \}$
\item $ \Sigma = \{\vartextvisiblespace, a, b, c, \rhd \}$
\item s es el estado inicial
\item $ H = \{ h, e \}$
\item y su función de transción $\delta$ viene definida de la siguiente manera:
  \hfill \break
  \begin{center}
    \begin{tabular}{|| c c c c ||}
      \hline
      Estado & Símbolo & Acción & N. Estado \\ [0.5ex]
      \hline\hline
      s & \vartextvisiblespace & $\rightarrow$ & $q_0$ \\
      \hline
      $q_0$ & \vartextvisiblespace & $\rightarrow$ & $q_0$ \\
      \hline
      $q_0$ & a & $\rightarrow$ & $q_1$ \\
      \hline
      $q_0$ & c & $\rightarrow$ & $q_7$ \\
      \hline
      $q_1$ & a & $\rightarrow$ & $q_1$ \\
      \hline
      $q_1$ & c & $\rightarrow$ & $q_1$ \\
      \hline
      $q_1$ & b & $\rightarrow$ & $q_2$ \\
      \hline
      $q_2$ & b & $\rightarrow$ & $q_2$ \\
      \hline
      $q_2$ & \vartextvisiblespace & $\leftarrow$ & $q_3$ \\
      \hline
      $q_3$ & b & \vartextvisiblespace & $q_4$ \\
      \hline
      $q_4$ & b & $\leftarrow$ & $q_4$ \\
      \hline
      $q_4$ & c & $\leftarrow$ & $q_4$ \\
      \hline
      $q_4$ & \vartextvisiblespace & $\leftarrow$ & $q_4$ \\
      \hline
      $q_4$ & a & $\leftarrow$ & $q_5$ \\
      \hline
      $q_5$ & a & $\leftarrow$ & $q_5$ \\
      \hline
      $q_5$ & \vartextvisiblespace & $\rightarrow$ & $q_6$ \\
      \hline
      $q_6$ & a & \vartextvisiblespace & $q_0$ \\
      \hline
      $q_7$ & \vartextvisiblespace & \vartextvisiblespace & $h$ \\
      \hline
      \_ & \_ & \vartextvisiblespace & $e$ \\
      \hline
    \end{tabular}
  \end{center}
\end{itemize}

\hfill \break
Para la segunda parte del ejercicio, definimos la MT $M_2$ de la
siguiente manera:

$ M_2 = < K, \Sigma, s, H, \delta >$

donde

\begin{itemize}
\item $ K = \{ s, q_{0}, q_{1}, q_{2}, q_{3}, q_{4}, q_{5}, q_{6}, q_{7}, q_{8}, q_{9}, q_{10}, h, e \}$
\item $ \Sigma = \{\vartextvisiblespace, a, \rhd \}$
\item s es el estado inicial
\item $ H = \{ h, e \}$
\item y su función de transción $\delta$ viene definida de la siguiente manera:
  \begin{center}
    \begin{tabular}{|| c c c c ||}
      \hline
      Estado & Símbolo & Acción & N. Estado \\ [0.5ex]
      \hline\hline
      s & \vartextvisiblespace & $\rightarrow$ & $q_0$ \\
      \hline
      $q_0$ & \vartextvisiblespace & $\rightarrow$ & $q_8$ \\
      \hline
      $q_0$ & a & $\rightarrow$ & $q_1$ \\
      \hline
      $q_1$ & a & $\rightarrow$ & $q_1$ \\
      \hline
      $q_1$ & \vartextvisiblespace & $\rightarrow$ & $q_2$ \\
      \hline
      $q_2$ & a & $\rightarrow$ & $q_2$ \\
      \hline
      $q_2$ & \vartextvisiblespace & $\leftarrow$ &$q_3$ \\
      \hline
      $q_3$ & a & \vartextvisiblespace & $q_4$ \\
      \hline
      $q_4$ & \vartextvisiblespace & $\leftarrow$ & $q_5$ \\
      \hline
      $q_5$ & a & $\leftarrow$ & $q_5$ \\
      \hline
      $q_5$ & \vartextvisiblespace & $\leftarrow$ & $q_6$ \\
      \hline
      $q_6$ & a & $\leftarrow$ & $q_6$ \\
      \hline
      $q_6$ & \vartextvisiblespace & $\rightarrow$ & $q_7$ \\
      \hline
      $q_7$ & a & \vartextvisiblespace & $q_0$ \\
      \hline
      $q_8$ & \vartextvisiblespace & $\rightarrow$ & $q_9$ \\
      \hline
      $q_8$ & a & a & $q_0$ \\
      \hline
      $q_9$ & \vartextvisiblespace & $\rightarrow$ & $q_{10}$ \\
      \hline
      $q_{10}$ & \vartextvisiblespace & \vartextvisiblespace & $h$ \\
      \hline
      \_ & \_ & \vartextvisiblespace & $e$ \\
      \hline
    \end{tabular}
\end{center}
\end{itemize}


\end{document}
