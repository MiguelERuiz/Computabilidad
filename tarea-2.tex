%% Preamble

\documentclass[11pt, a4paper, titlepage]{article}
\usepackage[utf8x]{inputenc}    % UTF-8 encoding
\usepackage[spanish]{babel} % culturallydetermined typographical (and other) rules

%% Fonts
\usepackage{fourier}
\usepackage[scaled=0.83]{helvet} %% Helvetica queda demasiado grande
% \renewcommand{\ttdefault}{txtt}
% \usepackage[scaled=0.9]{inconsolata}
\usepackage[scaled=0.75]{beramono}

\usepackage{hyperref}
\usepackage{parskip}            % Paragraph allingment
\usepackage{color}

% Macros for comments and todos
\usepackage[textsize=footnotesize, textwidth=2\marginparwidth]{todonotes}
\newcommand{\mr}[1]{\todo{MR: #1}}

%% Exercises
\usepackage[printsolution=true]{exercises}

%% Drawing Turing machines (automata in general)
\usepackage{tikz, wasysym}
\usetikzlibrary{automata,positioning}


%% Drawing diagrams
\usepackage[all]{xy}

%% visible white space
\newcommand\vartextvisiblespace[1][.5em]{%
  \makebox[#1]{%
    \kern.07em
    \vrule height.3ex
    \hrulefill
    \vrule height.3ex
    \kern.07em
  }% <-- don't forget this one!
}

%% End Preamble

%% Begin document

\begin{document}

\title{Tarea 2}
\date{17 de Febrero de 2018}
\maketitle
\newpage



\section*{Ejercicio 3.15}
\subsection*{Solución}


Para saber si el conjunto enunciado es o no decidible, vamos a hacer uso de las
deficiones que tienen que ver con definir un lenguaje:

Una MT \textit{M} decide un lenguaje \textit{L} si $\forall x \in \Sigma^*:$
\begin{itemize}
  \item Si x $\in \textit{L}, \textit{M}$ termina en Y.
  \item Si x $\notin \textit{L}, \textit{M}$ termina en N.
\end{itemize}

Un lenguaje es decidible o recursivo si $\exists \textit{M}$ que lo decide.


Por tanto, para poder demostrar que el conjunto \textit{B} es decidible, tenemos
que encontrar una MT \textit{M} que lo decida. La solución que propongo es la
siguiente:
\pagebreak

%% \[
%% \xymatrix{
%%   >R \ar[r]^{m} \ar[d]^{x} \ar@/_/[drr]^{\vartextvisiblespace} & R_{\overline{m}} \ar[r]^{\vartextvisiblespace, x} & R \ar[r]^{n} \ar[d]^{\vartextvisiblespace,x} & R_{\overline{n}} \ar[r]^{\vartextvisiblespace, x} & L_{\overline{x}} x L_{\vartextvisiblespace} L_{\overline{x}} x L_{\rhd} \ar@/_5pc/[llll] \\
%%   Y && N
%% }
%% \]

\[
\xymatrix{
  >R \ar[rr]^(.33){m} \ar@/_/[drr]^{\vartextvisiblespace} & & R_{\vartextvisiblespace}L_{m}xL_{\vartextvisiblespace}L_{m}L_{\rhd} \ar@/_2pc/[ll] \\
  &&N
}
\]


%% \[
%% \xymatrix{
%%   >R \ar[rr]^{\overline{a}} \ar@/_/[drr]^{b, \vartextvisiblespace } & & b \ar@/_2pc/[ll] \\
%%   &&L_{\rhd}
%% }
%% \]


%% \[
%% \xymatrix{ Hom_{R}(A,B) \ar[r]^{F(f)} \ar@/_2pc/[rr]_{F(hf), F(g)}  & Hom_{R}(A,C) \ar[r]^{F(h)} & Hom_{R}(A,D)}
%% \]


\section*{Ejercicio 3.20}
\subsection*{Solución}

\end{document}
