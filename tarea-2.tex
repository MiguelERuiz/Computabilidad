%% Preamble

\documentclass[11pt, a4paper, titlepage]{article}
\usepackage[utf8x]{inputenc}    % UTF-8 encoding
\usepackage[spanish]{babel} % culturallydetermined typographical (and other) rules

%% Fonts
\usepackage{fourier}
\usepackage[scaled=0.83]{helvet} %% Helvetica queda demasiado grande
% \renewcommand{\ttdefault}{txtt}
% \usepackage[scaled=0.9]{inconsolata}
\usepackage[scaled=0.75]{beramono}

\usepackage{hyperref}
\usepackage{parskip}            % Paragraph allingment
\usepackage{color}

% Macros for comments and todos
\usepackage[textsize=footnotesize, textwidth=2\marginparwidth]{todonotes}
\newcommand{\mr}[1]{\todo{MR: #1}}

%% Exercises
\usepackage[printsolution=true]{exercises}

%% Drawing Turing machines (automata in general)
\usepackage{tikz, wasysym}
\usetikzlibrary{automata,positioning}


%% Drawing diagrams
\usepackage[all,cmtip]{xypic}

%% visible white space
\newcommand\vartextvisiblespace[1][.5em]{%
  \makebox[#1]{%
    \kern.07em
    \vrule height.3ex
    \hrulefill
    \vrule height.3ex
    \kern.07em
  }% <-- don't forget this one!
}

%% End Preamble

%% Begin document

\begin{document}

\title{Tarea 2}
\date{17 de Febrero de 2018}
\maketitle
\newpage



\section*{Ejercicio 3.15}
\subsection*{Solución}


Para saber si el conjunto enunciado es o no decidible, vamos a hacer uso de las
deficiones que tienen que ver con definir un lenguaje:

Una MT \textit{M} decide un lenguaje \textit{L} si $\forall x \in \Sigma^*:$
\begin{itemize}
  \item Si x $\in \textit{L}, \textit{M}$ termina en Y.
  \item Si x $\notin \textit{L}, \textit{M}$ termina en N.
\end{itemize}

Un lenguaje es decidible o recursivo si $\exists \textit{M}$ que lo decide.


Por tanto, para poder demostrar que el conjunto \textit{B} es decidible, tenemos
que encontrar una MT \textit{M} que lo decida. La solución que propongo es la
siguiente: \footnote{La última acción sería retroceder hasta encontrar el
carácter de inicio de la cinta $\rhd$, pero no he sido capaz de dibujar bien el
diagrama.}

\[
\xymatrix{
  >R \ar[rr]^{m} \ar[d]^{x} \ar@/_/[rrd]^{\vartextvisiblespace} & & R_{\vartextvisiblespace}R \ar[d]^{x,\vartextvisiblespace} \ar[rr]^(.33){n} && R_{\vartextvisiblespace}L_{n}xL_{\vartextvisiblespace}L_{m}xL_{\rhd} \ar@(ul,ur)[llll] \\
  Y &&N
}
\]

\section*{Ejercicio 3.20}
\subsection*{Solución}

\end{document}
