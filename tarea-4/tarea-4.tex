%% Preamble

\documentclass[11pt, a4paper, titlepage]{article}
\usepackage[utf8x]{inputenc}    % UTF-8 encoding
\usepackage[spanish]{babel} % culturallydetermined typographical (and other) rules

%% Fonts
\usepackage{fourier}
\usepackage[scaled=0.83]{helvet} %% Helvetica queda demasiado grande
% \renewcommand{\ttdefault}{txtt}
% \usepackage[scaled=0.9]{inconsolata}
\usepackage[scaled=0.75]{beramono}

\usepackage{hyperref}
\usepackage{parskip}            % Paragraph allingment
\usepackage{color}

% Macros for comments and todos
\usepackage[textsize=footnotesize, textwidth=2\marginparwidth]{todonotes}
\newcommand{\mr}[1]{\todo{MR: #1}}

%% Exercises
\usepackage[printsolution=true]{exercises}

%% Math symbols
\usepackage{amssymb}
%% End Preamble

%% Begin document

\begin{document}

\title{Tarea 4}
\date{8 de marzo de 2018}
\maketitle

El punto de vista que presento\footnote{\url{https://journals.openedition.org/philosophiascientiae/772}}
es el de la profesora Carol E. Cleland, en el cual argumenta en contra de la
Tesis de Church, partiendo de la definición de un procedimiento superior a las
Máquinas de Turing (demoninados ``mundane procedures'') que tienen la capacidad
de calcular funciones que no podría una MT. Con esto Cleland no quiere dar por
falsa la tesis de Church, sino enseñar que los límites de la computación son
físicos y no lógicos.

Lo que distingue a los ``mundane procedures'' de las Máquinas de Turing es que se
pueden aplicar a procesos físicos y devuelven resultados empíricos, y además, el
resultado que devuelven es un resultado causal. Así pues, estos procedimientos
solo pueden ejecutarse bajo unas ciertas condiciones físicas (un contexto físico
tal y como se menciona en este resumen), por contra, tal y como argumenta Cleland,
las Máquinas de Turing no devuelven resultados causales sino formales, y ese
argumento le sirve para decir que la Tesis de Church no es válida, puesto
que los ``mundane procedures'' calculan funciones que una Máquina de Turing no podría
calcular.

Un argumento que se da es que una Máquina de Turing no podría calcular la función
identidad definida en el conjunto de los números reales:\\
\begin{equation}
  i \colon \mathbb{R} \to \mathbb{R}
\end{equation}

Partiendo de que haya una representación enumerable de números naturales para los
números reales, una MT nunca dejaría de leer los números que lo componen y no podría
devolver el número que se desea representar.


Personalmente pienso que es una buena argumentación para poner en duda la veracidad
de la Tesis de Church, pero las demostraciones que se dan son demasiado rebuscadas
como para pensar que es un punto de vista correcto.
\end{document}
